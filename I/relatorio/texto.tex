\begin{titlepage}
  \newgeometry{top=2.8cm,bottom=1cm,left=3cm,right=3cm}
  \normalfont\centering

  \normalsize
  \textsc{ISCTE-IUL}\\[-2pt]
  \textsc{Licenciatura em Ciência de Dados}\\
  \vspace{0.5cm}

  \Huge
  \rule[0.3cm]{\linewidth}{0.5pt}
  Trabalho Individual I \\
  \vspace{0.5cm}
  \large
  Exercício realizado no âmbito da Unidade Curricular de Optimização Heurística do 2º ano da Licenciatura em Ciência de Dados
  \rule[-0.5cm]{\linewidth}{1pt}\\

  % author
  \vspace{1.2cm}
  {\LARGE 
    André Plancha, 105289 \\
    \email{Andre\_Plancha@iscte-iul.pt} \\
  }
  % table of contents
  %\rule[-0.5cm]{\linewidth}{2pt}\\
  \normalsize
  \vspace{3cm}
  {
    \let\clearpage\relax
    \tableofcontents
  }

  % \restoregeometry

\end{titlepage}

\renewcommand{\d}[2]{d_{#1}^{#2}}
\newcommand{\p}[2]{p_{#1}^{#2}}
\newcolumntype{R}{>{$}r<{$}}
\newcolumntype{C}{>{$}c<{$}}
\newcolumntype{L}{>{$}l<{$}}

\section{a)}
Sendo $D_i$ o nível de produção do doce $i$ em milhares de \unit{\kilogram}, $d_i^+$ o desvio que significa o valor em excesso para atingir a meta $i$, $d_i^-$ o desvio que significa o valor em falta para atinger a meta $i$, e $p_i^\pm$ o peso associado ao desvio $d_i^\pm$, meu modelo será o seguinte:

\begin{tabular}[t]{l r}
  minimizar & $f(\d1-, \d2-, \d2+, \d3+; \p1-, \p2-, \p2+, \p3+)$ \\
  sujeito a & \\
            &
  %12D_1 + 9D_2 + 5D_3 + \d1-       \geq 125 \\
  %5D_1 + 2D_2 + 4D_3 + \d2- - \d2+ = 60 \\
  %5D_1 + 5D_2 + 8D_3 - \d3+        \leq 55
  %D_1 \leq 6
  %D_2 \geq 2
  %D_3 \geq 1
  \begin{tabular}[t]{R C R C R C R C R C R C L}
    12D_1  & + & 9D_2  & + & 5D_3  & + & \d1-  &         &          & \geq &125          \\
    5D_1   & + & 2D_2  & + & 4D_3  & + & \d2-  & -       & \d2+     & =    &60           \\
    5D_1   & + & 5D_2  & + & 8D_3  &   &       & -       & \d3+     & \leq &55           \\
    D_1    &   &       &   &       &   &       &         &          & \leq &6            \\
           &   &  D_2  &   &       &   &       &         &          & \geq &2            \\
           &   &       &   &  D_3  &   &       &         &          & \geq &1            \\
           &   &       &   &       &   &       & d_i^\pm,&  D_i     & \geq &0            \\
           &   &       &   &       &   &       &         &  d_2^\pm & \in  &\mathcal{Z}  \\

  \end{tabular}
 \end{tabular}\\
Em outros termos, o nosso objetivo vai ser minimizar uma combinação dos desvios pesados apropriada, com o objetivo de não derivar muito das 3 metas, cumprindo também as restrições de produções mínimas e máximas dos vários doces. As 3 primeiras restrições representam as metas do modelo, sendo a primeira uma meta de lucro mínimo, a segunda de uma meta de manter o atual número de empregados, e a terceira de uma meta de investimento máximo; estas estão representadas de acordo com a produção de cada doce que cada meta contribui para. Ou seja, por exemplo, o lucro vai ser:
$$
\text{lucro} = 12D_1 + 9D_2 + 5D_3 + \d1-
$$,
sendo $D_i$ a produção do doce $i$ em milhares de \unit{\kilogram} e $\d1-$ o desvio negativo da meta de lucro. Se $\d1-$ for $0$, então a meta de lucro é cumprida.

As seguintes 3 metas representam o máximo de produção do doce $D_1$, e os mínimos de produção dos doces $D_2$ e $D_3$, em milhares de \unit{\kilogram}. A penúltima restriçã indica a não negatividade da produção e dos desvios, e a última restrição indica que o desvio da meta de manter o atual número de empregados tem de ser um número inteiro, sendo que o número de empregados é um número inteiro. Esta última restrição indica também que $5D_1 + 2D_2 + 4D_3$ vai ser inteiro também. 

Antes de apresentar um exemplo de função objetivo (a que vai ser aplicada em \ref{sec:b}); o enunciado refere os pesos de penalização de cada desvio. Enquanto que $\p1- = 5$, e $p3+ = 5$ são pesos explícitos e diretos, os pesos dados pelo agente de decisão para a segunda meta são para cada 5 trabalhadores. Como este é um problema linear, os pesos vão ser transformados para 1 trabalhador invés. Ou seja:
$$
\p2- = \frac{4}{5} = 0,8 \land \p2+ = \frac{10}{5} = 2
$$

Um exemplo de função objetivo que pode ser aplicada neste modelo é a soma dos desvios pesados, ou seja:

\section{b)} \label{sec:b}